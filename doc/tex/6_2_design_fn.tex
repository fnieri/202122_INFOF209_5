\subsection{Fonctionnement du système}
        \subsubsection{Inscription}
        \addimg{img/6_Reg.eps}{width=\linewidth}{Register process}{regSeq}
        \subsubsection{Connexion}
        \addimg{img/6_Login.eps}{width=\linewidth}{Login process}{loginSeq}
        \subsubsection{Chat}
        \addimg{img/6_ChatSequence.eps}{width=\linewidth}{Chat process}{chatSeq}
        \subsubsection{Calcul du ELO}
        Pour comprendre comment fonctionne le matchmaking, il nous faut d'abord comprendre comment fonctionne le système de classement ELO, qui se résume en une phrase:
        \textit{"Le classement Elo est un système d’évaluation comparatif du niveau de jeu des joueurs d’échecs, de go ou d’autres jeux en un contre un."}\footnote{Wikipedia, Classement ELO} \\
        L'ELO est différent de modalité en modalité. \\
        Pour le calculer, nous n'allons pas développer un nouvel algorithme. Nous allons plutôt nous baser sur cet article \footnote{\href{https://metinmediamath.wordpress.com/2013/11/27/how-to-calculate-the-elo-rating-including-example/}{Metin's Media and Math, How To Calculate the Elo-Rating }}.
        Puisque le jeu Quoridor peut être joué par quatre joueur, il nous faut adapter le calcul afin que celui-ci reste équilibré. \\
        Supposons alors une première situation pour une partie en 1v1 avec les joueurs 1 et 2 dans l'ELO:
    \begin{center}
        r(1) = 1500 et r(2) = 1753
    \end{center}
    Nous allons tout d'abord calculer le \textit{transformed score} R(n) comme suit:
    \begin{equation}
        \begin{split}
            & R(1) = 10^{\frac{r(1)}{400}} = 5623,413 \\
            & R(2) = 10^{\frac{r(2)}{400}} = 24126,815 \\
        \end{split}
    \end{equation}

    Ensuite, nous calculons le \textit{expected score} E(n):
    \begin{equation}
        \begin{split}
            & E(1) = R(1) / ( R(1) + R(2)) = 5623,413 / (5623,413 + 24126,815) = 0,189 \\
            & E(2) = R(2) / ( R(1) + R(2)) = 24126,815 / (5623,413 + 24126,815) = 0,8109 \\
        \end{split}
    \end{equation}

    Définissons le \textit{actual score} pour les 2 joueurs:
    \begin{center}
        S(n) = 1 s'il gagne / 0 s'il perd
    \end{center}

    Calculons l'ELO final r'(n):
    Pour cela nous devons définir K: \\
    \textit{"This is called the K-factor and basically a measure of how strong a match will impact the players’ ratings.
    If you set K too low the ratings will hardly be impacted by the matches and very stable ratings (too stable) will occur.
    On the other hand, if you set it too high, the ratings will fluctuate wildly according to the current performance.
     Different organizations use different K-factors, there’s no universally accepted value. In chess the ICC uses a value of K = 32".}\footnote{\footnote{\href{https://metinmediamath.wordpress.com/2013/11/27/how-to-calculate-the-elo-rating-including-example/}{Metin's Media and Math, How To Calculate the Elo-Rating }}} \\
    Ici, nous supposons que le joueur 1 ait gagné.
    \begin{equation}
        \begin{split}
           & r'(1) = r(1) + K \cdot (S(1) - E(1)) = 1500 + 32 \cdot (1 - 0,189) = 1525,952 \\
           & r'(2) = r(2) + K \cdot (S(2) - E(2)) = 1753 + 32 \cdot (0-0,8109) = 1727,0512 \\
        \end{split}
    \end{equation}

    Les ELO finaux des deux joueurs seront :
    \begin{center}
        r(1) = 1526, +26 \ r(2) = 1727, -26
    \end{center}

    Supposons maintenant la partie se déroule entre 4 joueurs, 1, 2, 3 et 4:
    \begin{center}
        r(1) = 1700, r(2) = 1900, r(3) = 1500, r(4) = 1600
    \end{center}
    Comme avant, calculons les \textit{transformed scores}:
        \begin{equation}
            \begin{split}
                & R(1) = 10^{\frac{1700}{400}} = 17782,794 \\
                & R(2) = 10^{\frac{1900}{400}} = 56243,1325 \\
                & R(3) = 10^{\frac{1500}{400}} = 5623, 413  \\
                & R(4) = 10^{\frac{1600}{400}} = 10000 \\
            \end{split}
        \end{equation}
        Pour les \textit{expected scores} E(n) de chaque joueur, nous considérerons la moyenne pondérée des "transformed scores" R(m) des autres joueurs.
        Nous voulons que la moyenne soit pondérée afin de tenir compte de la différence entre les ELO des joueurs. Une disparité plus grande affectera
        de manière différente l'ELO final dans les cas d'une victoire ou d'une défaite.  \\
        Nous procéderons comme suit: \\
        Pour deux joueurs n et m: $\varepsilon = $ r(n) - r(m)  \\
        Nous considérons $< -100 \text{ et } > 100$ comme un écart significatif
            $$ \omega_{n, m} =
            \begin{cases}
                1 + \varepsilon \cdot 10^{-3}, & \text{si $\varepsilon \le -100$} \\
                1 - \varepsilon \cdot 10^{-3}, & \text{si $\varepsilon \ge 100$} \\
                1, & \text{si $-100 < \varepsilon < 100$ }\\
            \end{cases} $$

        E(1) sera alors, avec J le nombre de joueurs adversaires (3):
        \begin{equation}
            \begin{split}
                & E(1) = R(1) / [R(1) + (\dfrac{R(2) \cdot \omega_{1, 2} + R(3) \cdot \omega_{1, 3} + R(4) \cdot \omega_{1, 4}}{J})] = 0,394 \\
            \end{split}
        \end{equation}
        E(2), E(3) et E(4) se calculeront comme suit:
        \begin{equation}
            \begin{split}
                & E(2) = R(2) / [R(2) + (\dfrac{R(1) \cdot \omega_{2, 1} + R(3) \cdot \omega_{2, 3} + R(4) \cdot \omega_{2,4} }{J})] = 0,872 \\
                & E(3) = R(3) / [R(3) + (\dfrac{R(1) \cdot \omega_{3, 1} + R(2) \cdot \omega_{3, 2} + R(4) \cdot \omega_{3,4} }{J})] = 0,131 \\
                & E(4) = R(4) / [R(4) + (\dfrac{R(1) \cdot \omega_{4, 1} + R(2) \cdot \omega_{4, 2} + R(3) \cdot \omega_{4,3} }{J})] = 0,232 \\
            \end{split}
        \end{equation}
        S(n) sera défini comme avant : 1 si le joueur a gagné, 0 s'il a perdu.

        Le calcul du ELO final r'(n) s'effectue de la même manière:
        \begin{equation}
            \begin{split}
                & r'(1) = r(1) + K \cdot (S(1) - E(1)) = 1719 \\
                & r'(2) = r(2) + K \cdot (S(2) - E(2)) = 1872 \\
                & r'(3) = r(3) + K \cdot (S(3) - E(3)) = 1495 \\
                & r'(4) = r(4) + K \cdot (S(4) - E(4)) = 1592 \\
            \end{split}
        \end{equation}

        L'ELO final pour chaque joueur correspondra alors à r'(n). \\
        Nous pouvons constater que les ELO finaux sont raisonnables si mis en rapport avec la disparité de l'ELO initiale. Les joueurs 3 et
        4 ne voient pas leur ELO diminuer d'une grande marge. En revanche, le joueur 2 qui possédait auparavant l'ELO le plus haut, constate une diminution de 28.
