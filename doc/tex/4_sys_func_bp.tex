\section{Besoins système fonctionnels}
\subsection{Connexion}

L'accès au fonctionnalités du système ne doit être accordé
qu'aux utilisateurs ayant fourni un nom d'utilisateur (ou
\emph{username}) et un mot de passe (ou \emph{password})
corrects. Ceux-ci sont corrects s'ils correspondent
 à une entrée dans la base de données.

Le système doit donc pouvoir confirmer les identifiants
grâce à une requête au serveur depuis
la machine de l'utilisateur.
Si la partie client ne parvient pas à établir une connexion
avec la partie serveur,
le système doit pouvoir en informer
l'utilisateur et refuser toute tentative de connexion.
% Tous deux doivent être composés de caractères
% alphanumériques.

Le système doit pouvoir donner l'occasion à l'utilisateur d'entrer
ses informations grâce à un formulaire ou une demande
séquentielle. Si les informations entrées ne sont pas
correctes, le système doit pouvoir communiquer à
l'utilisateur un message d'erreur utile et lui laisser la
possibilité d'essayer à nouveau.

% Si l'utilisateur ne se souvient plus de son nom
% d'utilisateur ou de son mot de passe, le
% système doit pouvoir lui fournir un moyen de regagner accès
% à son compte.

% \subsubsection{Regain d'accès au compte utilisateur}

% Un courriel de récupération contenant le nom d'utilisateur et un
% hyperlien renvoyant vers une page permettant le changement
% du mot de passe doit pouvoir être envoyé
% dans le cas d'un
% oubli d'identifiants.

% Le mot de passe étant chiffré par un processus irréversible dans la base
% de données, seul un nouveau mot de passe permet un regain
% d'accès.

% Le processus de changement de mot de passe doit rendre
% l'ancien mot de passe invalide et rendre le nouveau mot de
% passe l'unique clé d'accès au compte associé.

\subsection{Gestion des comptes}

Un compte doit être identifié grâce à une adresse courriel
et un nom d'utilisateur. Chaque compte doit être unique
et par conséquent, deux comptes différents ne peuvent pas avoir
les mêmes identifiants. Le système doit pouvoir accéder aux
identifiants séparément ou en lot.

Le mot de passe, bien qu'essentiel au compte, ne doit pas
faire
partie de son identité. Deux comptes différents
peuvent avoir le même mot de passe.

Les informations relatives aux comptes des utilisateurs
doivent être stockées dans une base de données centrale et
accessibles depuis chaque client \emph{via} Internet.

\subsubsection{Création de compte}

Si un visiteur du système ne possède pas de compte
utilisateur, le système doit pouvoir lui donner la
possibilité d'en créer un.

Durant ce processus, le système doit pouvoir demander au
visiteur~: un nom d'utilisateur et un
mot de passe. Le système doit s'assurer que le
premier champ n'existe pas déjà dans la base de données.
Si c'est le cas, le système doit refuser la demande
d'inscription et en informer le visiteur.

% Si les identifiants sont tous deux uniques, le système doit
% pouvoir envoyer un courriel de confirmation à l'adresse
% courriel fournie. Le compte ainsi créé ne sera actif qu'à
% partir de la confirmation de l'adresse courriel par le
% visiteur et deviendra dès lors un utilisateur.

\subsubsection{Suppression de compte}

Le système doit donner l'opportunité à tout utilisateur de
supprimer son compte. Ce processus de suppression se solde
par une suppression définitive et irrévocable du compte
associé dans la base de données.
Les identifiants qui étaient utilisés par ce
dernier doivent être de nouveau disponibles pour de
nouvelles inscriptions.

Le système ne doit supprimer en aucun cas un compte
utilisateur sans la confirmation de son titulaire.

\subsection{Menu principal}

Le menu principal du système doit pouvoir donner accès aux
actions suivantes à tout utilisateur authentifié~:

\begin{itemize}
    \item jouer~;
    \item consulter les classements~;
    \item discuter avec ses amis~;
    \item modifier ses informations personnelles.
\end{itemize}

\subsection{Interactions entre utilisateurs}

Le système doit donner la possibilité à chaque utilisateur
authentifié de gérer une liste d'amis. L'utilisateur doit
pouvoir ajouter des amis et en supprimer. L'ajout se
fait au moyen de demandes d'ami que tout utilisateur peut
envoyer à tout autre utilisateur à condition qu'il connaisse
son nom d'utilisateur.

Ces listes d'amis sont stockées dans la base de données et
propres à chaque utilisateur, personne d'autre que
l'utilisateur authentifié ne devrait y avoir accès.

\subsubsection{Discuter avec ses amis}


Le système doit donner la possibilité à tout utilisateur
authentifié de discuter avec ses amis par messages écrits
dans des fils de discussion privés.

% Le système doit
% également offrir la possibilité aux utilisateurs de créer
% des groupes de discussion où plusieurs utilisateurs
% peuvent participer. Les participants doivent pouvoir être
% ajoutés soit lors de la création du groupe, soit après
% celle-ci par tout participant du groupe grâce à un système
% d'invitations.

Tout message envoyé sur ces fils de discussion doit être
sauvegardé dans la base de données dès son envoi,
visible par tous les
participants et supprimable par son auteur. Les messages
doivent être persistants d'une session à l'autre.

Pour pouvoir supprimer les messages individuellement, ces
derniers doivent posséder un identificateur unique composé
notamment de l'heure d'envoi et de l'expéditeur.

\subsection{Consulter les classements}

Tout utilisateur doit avoir un score qui indique son niveau
au jeu. Ce score est modifié à l'issue de chaque partie
par le système d'évaluation \emph{ELO}.

Le système doit pouvoir, grâce à ce score, constituer
un classement des meilleurs utilisateurs. Ce classement doit
pouvoir être visible par tout utilisateur authentifié du système.

Ce classement doit être consultable et filtrable.
L'utilisateur doit avoir la possibilité d'afficher
soit totalité des utilisateurs, soit uniquement les utilisateurs
amis.

\subsection{Modifier ses informations personnelles}

Le système doit fournir la possibilité à tout utilisateur de
modifier ses informations personnelles. Cela inclut 
le nom d'utilisateur et le mot de passe.
La modification de chacune de ces informations doit être
définitive et la nouvelle valeur doit être
sauvegardée dans la base de données.

\subsection{Jouer}

Pour jouer, l'utilisateur doit initialiser une recherche
d'adversaire pour le mode de
jeu qu'il a choisi. Cette étape est ce qu'on appelle le
\emph{matchmaking}. Le système doit s'assurer que les
utilisateurs choisis pour une partie aient choisi le même
mode de jeu.
% Le système doit s'assurer que lorqu'un
% match est prêt à être lancé, les deux utilisateurs choisis soient
% toujours connectés.

\subsubsection{Matchmaking}

Lorqu'un joueur entre dans le matchmaking, il doit être
placé dans une file d'attente à priorité.
La priorité de chaque joueur est définie selon la durée
écoulée
depuis son entrée dans le matchmaking. Cette durée est dite \emph{temps d'attente}.
Pour séparer les modes de jeu, le système doit gérer une
file d'attente distincte pour chaque mode de jeu.

Le choix d'un adversaire adéquat doit ainsi dépendre de son temps
d'attente mais aussi, et surtout, de son \emph{ELO}.
À l'issue d'une situation optimale, les deux utilisateurs doivent
avoir un \emph{ELO} similaire et avoir eu un temps d'attente presque
nul.

Chaque utilisateur doit avoir la possibilité de voir le nombre d'utilisateurs
dans la file d'attente ainsi que son temps d'attente. Le
système doit lui donner la possibilité de quitter le
matchmaking.

Un utilisateur dans le matchmaking doit y être est enlevé
dès qu'il y a trouvé un adversaire, que sa connexion au
serveur n'est plus valide ou qu'il a quitté la file
d'attente de son plein gré.

\subsubsection{Partie de Quoridor}

Lors d'une partie, le système doit faire une distinction
entre les actions disponibles aux participants en tant que
\emph{joueurs}, et
celles disponibles en tant qu'\emph{utilisateurs}.

Un joueur est celui qui fait des actions sur le
plateau de jeu, alors qu'un utilisateur fait des
actions en-dehors du plateau de jeu. Le joueur ne peut agir
que lorsque c'est son tour de jouer alors que l'utilisateur, au
contraire, peut agir tout le temps.

Le système doit permettre aux participants de jouer,
chacun à leur tour, jusqu'à une victoire potentielle
de l'un

Lors d'une fin de partie, le système doit en informer les
participants, mettre à jour l'\emph{ELO} de chaque
utilisateur et renvoyer les utilisateurs vers le menu principal.


% Création d'une partie, matchmaking
% Gestion d'une partie, gestion murs, gestion pions, gestion
% score, gestion vainqueur
% Mise à jour du classement



