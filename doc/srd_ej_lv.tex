\documentclass[utf8]{article}

\usepackage[utf8]{inputenc}

\usepackage[parfill]{parskip}

\usepackage{amsmath}
\usepackage{amssymb}
\usepackage{amsfonts}
\usepackage{graphicx}
\usepackage{float}
\usepackage{algorithm}
\usepackage{algorithmicx}
\usepackage{algpseudocode}
\usepackage{fullpage}
\usepackage{hyperref}



%-----------------------------------------------------------

\newcommand{\up}[1]{\textsuperscript{#1}}
\newcommand{\down}[1]{\textsubscript{#1}}

\newcommand{\addcode}[3]{
    \begin{figure}[H]
        \centering
        \lstinputlisting[language=#2, caption=\textattachfile{#1}{#1}, label=#3]{#1}
    \end{figure}
}

\newcommand{\addimg}[4]{
    \begin{figure}[H]
        \centering
        \includegraphics[#2]{#1}
    \caption{#3}
    \label{#4}
    \end{figure}
}

%------------------------------------------------------
% -----------------------------------------------------

\begin{document}

\tableofcontents

\newpage

% -----------------------------------------------------

\section{Notre jolie game diagramme (prend moi ça Boris)}

Le jeu est divisé en deux plateaux. Le plateau mur (Wall board) contient les murs. Alors que le plateau joueur contient les pions des
joueurs. Chaque pion a une couleur différente afin de permettre aux joueurs de se différencier l'un de l'autre. La classe board combine
chaque plateau, qui sera ensuite afficher par BoardPrinter dans MainWindow. 

Le PlayerBoard est divisé en cases. Chaque case peut accueillir au maximum un pion définit par la classe Player. Chaque Player peut
se déplacer sur le PlayerBoard en faisant appel à la classe PlayerAction, qui vérifie si le coup est valide.

Le WallBoard est quant à lui est divisé en couloir. Chaque couloir peut contenir un Wall. Un couloir a une largeur de 2 cells. L'utilisateur
peut placer un wall en faisant appel à WallAction qui va d'abord vérifier si l'action est valide.

\section{Partie 5: Besoin système non fonctionnels}
\subsection{Système d'exploitation}
Le programme doit être capable de tourner sous Linux, plus précisément avec la distribution Ubuntu 20.04 même s'il devrait savoir tourner
sur n'importe quelle autre distribution.
Ce serait potentiellement justifié de créer une version Windows et MacOS, mais ce n'est pas une priorité.

\subsection{Réseau}
Le programme va faire l'utilisation de socket pour permettre la communication entre le serveur et le client. 
Le client devra éventuellement faire des requêtes vers des APIs externes.

\subsection{Disponibilité}
Afin que le programme fonctionne correctement, il est primordial que le serveur soit disponible, c'est à dire existant et accessible. 

\subsection{Performance}
Étant donné que le programme ne nécessite pas une grande performance graphique et que les calculs de jeu sont effectués sur un serveur, le
programme est propice à avoir une bonne performance. Il faut s'assurer que les calculs exécutés sur le serveur soit effectués le plus
rapidement possible afin de renvoyé une réponse au client le plus rapidement possible. Le jeu doit pouvoir être joué rapidement, il faut
donc réduire au maximum le temps d'attente entre chaque coup.
Il faut aussi que le programme utilise 2 threads, un pour l'affichage graphique et un pour les requêtes asynchrones, afin d'éviter que le
programme ne sache plus réagir aux input du client.

\subsection{Capacité}
Le client doit avoir la capacité de lire des données du serveur et de les afficher à l'utilisateur. Il doit aussi être capable de gérer les
entrées de l'utilisateur et d'en faire part au serveur.

Le serveur doit avoir la capacité de gérer chaque calcul lié au jeu, par exemple vérifié qu'un coup est valide. Il doit aussi être capable
de manipuler dans une base de données le chat, les données en rapport au jeu, tel que l'état de la partie en cas de sauvegarde, et les 
données utilisateurs.

\subsection{Sécurité}
Afin d'être correctement sécurisé, le programme va encrypter les mots de passe avant des les enregistrer dans la base de données avec un
algorithme de hashing, tel que SHA-256. Une salt key sera aussi utilisée lors de l'encryption pour amélioré la sécurité.

Il peut aussi s'avérer utile d'établir une socket sécurisé à l'aide d'un certificat SSL. Cela permettrait entre autre de communiquer par HTTPS au
lieu que par HTTP.

\subsection{Robustesse}
Le client doit être capable de résister à tout input de l'utilisateur sans cracher.
Le serveur doit savoir gérer n'importe quel type de requêtes sans crash. Il doit aussi être capable de limiter ses actions au bon utilisateur.

\section{Partie 6: Design et fonctionnement du système}
\subsection{fonctionnement du système}
\subsubsection{Boucle du jeu: Client}
\begin{itemize}
    \item Flow of a game (use activity diagram made by Boris)

        À chaque tour, le client va envoyer une requête au serveur. Si le jeu est terminé, le serveur renverra un message au client pour que
        celui-ci puisse arrêter la boucle de jeu.
    \item Chat

        Le chat se met à jour à chaque fois que le serveur envoie un message au client. Et un player écris un message qui est envoyé au serveur.
\end{itemize}
\subsubsection{Boucle du jeu: Serveur}
\begin{itemize}
    \item Jeu

        Lorsque le client joue un coup, le serveur va vérifier que ce coup est valide et que la partie n'est pas fini. Il informera ensuite
        le client.

    \item Chat
    
        Le serveur doit informer les clients lorsque un message a été envoyé dans le chat.


\end{itemize}

\end{document}
